\documentclass{article}

\usepackage{fancyhdr}
\usepackage{extramarks}
\usepackage{amsmath}
\usepackage{amsthm}
\usepackage{amsfonts}
\usepackage{tikz}
\usepackage[plain]{algorithm}
\usepackage{algpseudocode}
\usepackage{placeins}
\usepackage{tcolorbox}

\usetikzlibrary{automata,positioning}

%
% Basic Document Settings
%

\topmargin=-0.45in
\evensidemargin=0in
\oddsidemargin=0in
\textwidth=6.5in
\textheight=9.0in
\headsep=0.25in

\linespread{1.1}

\pagestyle{fancy}
\lhead{\hmwkAuthorName}
\chead{\hmwkClass\: \hmwkTitle}
\rhead{\firstxmark}
\lfoot{\lastxmark}
\cfoot{\thepage}

\renewcommand\headrulewidth{0.4pt}
\renewcommand\footrulewidth{0.4pt}

\setlength\parindent{0pt}

\newtheorem{subproblem}{Subproblem}

\tcbuselibrary{theorems}

\newtcbtheorem[number within=section]{mytheo}{Definition}%
{colback=blue!5,colframe=blue!35!black,fonttitle=\bfseries}{th}

%
% Create Problem Sections
%

\newcommand{\enterProblemHeader}[1]{
    \nobreak\extramarks{}{Problem \arabic{#1} continued on next page\ldots}\nobreak{}
    \nobreak\extramarks{Problem \arabic{#1} (continued)}{Problem \arabic{#1} continued on next page\ldots}\nobreak{}
}

\newcommand{\exitProblemHeader}[1]{
    \nobreak\extramarks{Problem \arabic{#1} (continued)}{Problem \arabic{#1} continued on next page\ldots}\nobreak{}
    \stepcounter{#1}
    \nobreak\extramarks{Problem \arabic{#1}}{}\nobreak{}
}

\setcounter{secnumdepth}{0}
\newcounter{partCounter}
\newcounter{homeworkProblemCounter}
\setcounter{homeworkProblemCounter}{1}
\nobreak\extramarks{Problem \arabic{homeworkProblemCounter}}{}\nobreak{}

%
% Homework Problem Environment
%
% This environment takes an optional argument. When given, it will adjust the
% problem counter. This is useful for when the problems given for your
% assignment aren't sequential. See the last 3 problems of this template for an
% example.
%
\newenvironment{homeworkProblem}[1][-1]{
    \ifnum#1>0
        \setcounter{homeworkProblemCounter}{#1}
    \fi
    \section{Problem \arabic{homeworkProblemCounter}}
    \setcounter{partCounter}{1}
    \enterProblemHeader{homeworkProblemCounter}
}{
    \exitProblemHeader{homeworkProblemCounter}
}

%
% Homework Details
%   - Title
%   - Due date
%   - Class
%   - Section/Time
%   - Instructor
%   - Author
%

\newcommand{\hmwkTitle}{Past Paper May 2022}
\newcommand{\hmwkDueDate}{May 5, 2023}
\newcommand{\hmwkClass}{ASR}
\newcommand{\hmwkAuthorName}{\textbf{Patrick Tourniaire}}

%
% Title Page
%

\title{
    \vspace{2in}
    \textmd{\textbf{\hmwkClass:\ \hmwkTitle}}\\
    \normalsize\vspace{0.1in}\small{Completed\ on\ \hmwkDueDate}\\
    \vspace{3in}
}

\author{\hmwkAuthorName}
\date{}

\renewcommand{\part}[1]{\textbf{\large Part \Alph{partCounter}}\stepcounter{partCounter}\\}

%
% Various Helper Commands
%

% Useful for algorithms
\newcommand{\alg}[1]{\textsc{\bfseries \footnotesize #1}}

% For derivatives
\newcommand{\deriv}[1]{\frac{\mathrm{d}}{\mathrm{d}x} (#1)}

% For partial derivatives
\newcommand{\pderiv}[2]{\frac{\partial}{\partial #1} (#2)}

% Integral dx
\newcommand{\dx}{\mathrm{d}x}

% Alias for the Solution section header
\newcommand{\solution}{\textbf{\large Solution}}

% Probability commands: Expectation, Variance, Covariance, Bias
\newcommand{\E}{\mathrm{E}}
\newcommand{\Var}{\mathrm{Var}}
\newcommand{\Cov}{\mathrm{Cov}}
\newcommand{\Bias}{\mathrm{Bias}}

\begin{document}

\maketitle

\pagebreak


% Problem 1
\begin{homeworkProblem}
    
    % Sub-problem 1
    \begin{subproblem}
        What is the difference, if any, between pitch and fundamental frequency?
    \end{subproblem}

    \textbf{Answer}

    The fundemental frequency is closely related to pitch, which is defined as our perception of fundemental frequency. That is, the $F0$
    describes the actual physical phenomenon, whereas pitch describes how our ears and brains intrepret the signal, in terms of periodicity.


    % Sub-problem 2
    \begin{subproblem}
        When we hear the high voice of children, is it because of their shorter
        vocal tract, their shorter vocal fold, or both? Why?
    \end{subproblem}

    \textbf{Answer}

    When we hear the high-pitched voices of children, it is mainly due to the shorter length of their vocal tract, rather than the lenght of
    their vocal folds. The pitch of a voice is determined by the frequency of the vibrations produced by the vocal folds. When the vocal
    folds vibrate at a higher frequency, the resulting sound has a higher pitch. However, the vocal tract, which includes the mouth, throat,
    and nasal cavity, also plays a significant role in shaping the sound produced by the vocal folds.

    In children, the vocal tract is shorter and narrower than in adults, which results in a higher resonance frequency. This means that when
    the vocal folds vibrate, the resulting sound waves are amplified at a higher frequency, resulting in a higher-pitched voice. As children
    grow and their vocal tract elongates, their voices gradually deepen and become lower in pitch.
    
    So, to sum up, the high-pitched voices of children are primarily due to the shorter length of their vocal tract, which amplifies the
    higher-frequency vibrations produced by their vocal folds.


    % Sub-problem 3
    \begin{subproblem}
        In the ideal speech production model described in class, what is the
        connection, if any, between formants and fundamental frequency?
    \end{subproblem}

    \textbf{Answer}

    The fundemental frequency is the first frequency component of the glottal pulse, whereas formants are the resonance frequencies of the vocal
    tract. Which leads to the production and perception of certain phones, particularly vowels.

    
    % Sub-problem 4
    \begin{subproblem}
        Could we infer fundamental frequency from log Mel spectrograms? If
        so, how?
    \end{subproblem}

    \textbf{Answer}

    Yes, it is possible to infer the fundemental frequency (also know as $F0$ or pitch) from log Mel spectrograms is to use a technique called the
    autocorrelation method. The basic idea behind this method is to calculate the correlation between the spectogram and a delayed version of itself,
    and then identify the delay that results in the highest correlation. This delay corresponds to the period of the fundemental frequency, which can
    be used to calculate the $F0$.

    Another approach is to use a DNN based method to directly estimate $F0$ from the log Mel spectogram. This involves training a NN on a large dataset
    of audio recordings and their corresponding $F0$ values, so that the network learns to recognise patterns in the spectogram that are associated with
    different pitch values. Once the network is trained, it can be used to predict $F0$ for new spectograms.


    % Sub-problem 5
    \begin{subproblem}
        One of your friends taking the ASR course suggests we use a 3-state HMM
        to model the high (H) and low (L) of fundamental frequency contours. At
        each state, the HMM can emit a symbol H or a symbol L. You can find
        the transition probabilities and the emission probabilities in Tables 1 and 2,
        respectively. We only allow sequences that start at state 1 and end at state
        3. In other words, the prior probability is 1.0 for state 1 and 0.0 for others

        What is the joint probability of emitting HHLLL for the state sequence
        12223?
    \end{subproblem}

    \textbf{Answer}

    \begin{align}
        Q &= [ 1, 2, 2, 2, 3 ] \\
        X &= [ H, H, L, L, L ] \intertext{Using these parameters we can calculate the joint probability of the sequence in the HMM.}
        P(X, Q; \lambda) &= P(1) P(H | 1) P(2 | 1) P (H | 2) P (2 | 2) P (L | 2) P (2 | 2) P (L | 2) P (3 | 2) P (L | 3) \\
                         &= 1.0 \times 0.75 \times 0.80 \times 0.50 \times 0.60 \times 0.50 \times 0.60 \times 0.50 \times 0.40 \times 0.75 \\
                         &= 0.0081
    \end{align}

    
    % Sub-problem 5
    \begin{subproblem}
        What is the most likely state sequence given that we observe HHLLL?
    \end{subproblem}

    \textbf{Answer}

    By maximising the joint-probability, the most likely state sequency for the above observation would be $Q = [ 1, 1, 2, 3, 3 ]$.


    % Sub-problem 6
    \begin{subproblem}
        What is the marginal probability of emitting HHLLL
    \end{subproblem}

    \textbf{Answer}

    \textit{TOOD:} Calculate using forward probabilities.



\end{homeworkProblem}


\begin{homeworkProblem}
    
    % Sub-problem 1
    \begin{subproblem}
        
    \end{subproblem}

\end{homeworkProblem}



\end{document}
